
Over the years, there has been an increase in the mortality rate of hospital patients due to short-handed workers. Hospitals are understaffed, and the few available are usually overworked, leading to exhaustion and mental breakdown of healthcare workers. This inherently results in inefficient care and the death of patients. 

Among its various proposed use cases, this project aims to reduce the physical strain on healthcare workers by reducing their workload. Using an automated delivery system, routine and minute tasks like delivering medical supplies, meals, laboratory samples, laundry, reminders, etc. will be carried out. This should allow nurses and doctors to better spend their time doing more hands-on tasks.

The Medical Supply Delivery System (MSDS) seeks to harness the power of AI to create an autonomous robot model that integrates Algorithms, Odometry, Path Planning, Navigation, Mapping, and Localization powered by various sensors. 

This delivery system will be suitable for dynamic hospital environments, focusing on real-time navigation, obstacle avoidance, efficient task scheduling and execution.

\vspace{0.5cm}
\noindent\textbf{Keywords:} Robotics, Systems Engineering, Optimization, Automation, ROS, Autonomous